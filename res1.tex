\begin{answer}{1.2}
Rešitve, po vrsti, so:
    \begin{enumerate}
        \item $(A \lor B) \land \neg C$
        \item $(A \land B) \lor \neg (A \land B)$
        \item $\neg (A \land \neg C)$
        \item $\neg ((B \lor C) \land \neg A)$
    \end{enumerate}
\end{answer}
\begin{answer}{1.4}
 Nariši skico. Premica $q$ leži v ravnini, in jo določata $p$ in $r$.
\end{answer}
\begin{answer}{1.8}
Naj bo $A$ izjava: ``A je vitez", itd.
Iščemo tisto edino določilo $d$, za katerega je izjava $$A_1\inn B_1\inn C_1\inn D_1$$
pravilna, kjer je:

$A_1: A\cee (\neg D \inn \neg C)$

$B_1: B\cee (\neg A \inn \neg D\sledi \neg C)$

$C_1: C\cee (\neg B \sledi A)$

$D_1: D\cee (\neg E \sledi \neg C \inn \neg B)$


Ker bi pravilnostna tabela vsebovala 32 vrstic, rešimo nalogo
raje z analizo primerov.

\textbf{1.~primer: $A(d) = 1$.}
Zaradi $A_1$ je potem $D(d) = 0$ in $C(d) = 0$.

V izjavo $C_1$ vstavimo $A(d) = 1$ in $C(d) = 0$, dobimo:
$\neg(\neg B\sledi 1)$,

$\neg(B\ali 1)$,

$\neg 1$, to pa je nepravilna izjava.

Torej 1.~primer ni mogoč.
%
% torej $B(d) = 1$.
%
%V izjavo $B_1$ vstavimo $A(d) = 1$, $B(d) = 1$, $C(d) = 0$ in $D(d) = 0$, dobimo:
%$1\cee (0\inn 1\sledi 1)$, kar je pravilna izjava.
%
%V izjavo $D_1$ vstavimo znane vrednosti, dobimo:
%$\neg(\neg E \sledi 1 \inn 0)$
%
%$\neg(\neg E \sledi 0)$
%
%Sledi $\neg E(d) = 1$, torej $E(d) = 0$.
%

\textbf{2.~primer: $A(d) = 0$.}

Zaradi $A_1$ je bodisi $C(d) = 1$ ali pa $D(d) = 1$.

\textbf{2.1.: $C(d) = 1$}.

Zaradi $C_1$ je $\neg B \sledi 0$, torej je $\neg B = 0$ in posledično $B(d) = 1$.

V izjavo $B_1$ vstavimo $A(d) = 0$, $B(d) = 1$, $C(d) = 1$, dobimo:

$1 \inn \neg D\sledi 0$

$\neg D\sledi 0$

Sledi $\neg D = 0$ oz.~$D(d) = 1$.

Vstavimo v izjavo $D_1$ znane vrednosti:

$(\neg E \sledi 0 \inn 0)$

Sledi $E(d) = 1$.

\bigskip

\textbf{2.2.: $C(d) = 0$ in $D(d) = 1$}.

Iz izjave $B_1$ dobimo $B(d) = 1$.

Izjava $C_1$ pa je sedaj nepravilna:
$0 \cee (0\sledi 1)$.\qed

Torej so $B$, $C$, $D$ in $E$ vitezi, $A$ pa je oproda.

\end{answer}
\begin{answer}{1.9}
Označimo:

$A_1$: Mislim.

$A_2$: Sem.

$A_3$: Sklepam.

Zanima nas pravilnost implikacije

$$(A_1\sledi A_2)\inn(A_1\sledi A_3)\sledi(A_2\sledi A_3)$$

Pri določilu $A_1(d) = 0$, $A_2(d) = 1$, $A_3(d) = 0$ je ta implikacija nepravilna!
(Ne mislim, sem, ne sklepam.)
Torej je sklepanje napačno.

\end{answer}
\begin{answer}{1.10}
Označimo izjave:

$A_1$: Sem dojenček.

$A_2$: Obnašam se nelogično.

$A_3$: Sposoben sem ukrotiti krokodila.

$A_4$: Vreden sem spoštovanja.

$(A_1\sledi A_2)\inn (A_3\sledi A_4) \inn (A_2\sledi \neg A_4)\sledi (A_1\sledi \neg A_3)$

Pa recimo, da je sklep napačen. Tedaj obstaja določilo $d$, da velja
\begin{enumerate}[(1)]
  \item $(A_1(d)\sledi \neg A_3(d)) = 0$
  \item $(A_1(d)\sledi A_2(d)) = 1$
  \item $(A_3(d)\sledi A_4(d)) = 1$
  \item $(A_2(d)\sledi \neg A_4(d)) = 1$
\end{enumerate}
Torej je, zaradi (1), $A_1(d) = 1$ in $A_3(d) = 1$. Zaradi (2) je $A_2(d) = 1$.
Zaradi (4) je $A_4(d) = 0$. To pa je protislovje s (3).

Torej je sklepanje pravilno.\qed
\end{answer}
\begin{answer}{1.11}
Označimo izjave:

$K$: Morilka je kuharica.

$S$: Morilec je strežnik.

Š: Morilec je šofer.

H: Kuharica je zastrupila hrano.

B: Šofer je postavil bombo v avto.

\medskip
Ali je naslednja implikacija tavtologija?

$(K\ali S\ali$ \v S$)\inn(K\sledi H)\inn($Š$\sledi B)\inn(\neg H\inn \neg S)\sledi $Š

Dokazujemo direktno:
\bigskip
\begin{enumerate}
    \item $(K\ali S\ali $Š$)$\hfill (Predpostavka)
    \item $(K\sledi H)$\hfill (Predpostavka)
    \item $($Š$\sledi B)$\hfill (Predpostavka)
    \item $(\neg H\inn \neg S)$\hfill (Predpostavka)
    \item $\neg H$\hfill (4)
    \item $\neg S$\hfill (4)
    \item $\neg K$\hfill (2,5)
    \item Š\hfill (1,6,7)
\end{enumerate}

\end{answer}
\begin{answer}{1.12}

(i) Napiši pravilnostno tabelo. DNO: vzemi vrstice z enicami (poveži jih med sabo s konjunkcijo) in jih poveži med sabo z disjunkcijo
$(A\wedge B) \vee (A\wedge \neg B) \vee (\neg A \wedge B)$. KNO: vzemi vrstice z ni"clami (vzemi nasprotne vrednosti in jih poveži med sabo z disjunkcijo) in jih poveži med sabo s konjunkcijo $(A \vee B)$.

(ii) Podobno.
\end{answer}
\begin{answer}{1.14}
Napiši pravilnostno tabelo za osnovni izjave $A, B$ skupaj s (sestavljeno) izjavo $\mathcal{I}$. Iz nje razberi, da je pravilnostna tabela za $\mathcal{I}$ enaka
\begin{table}[ht!]
\centering
\begin{tabular}{c|c|c}
A & B & $\mathcal{I}$\\
\hline
1 & 1 & 0\\
1 & 0 & 1\\
0 & 1 & 1\\
0 & 0 & 1
\end{tabular}
\end{table}

Torej je $\mathcal{I} \Leftrightarrow \neg A \vee \neg B$ v KNO.
\end{answer}
\begin{answer}{1.16}

\begin{eqnarray*}
(A\Rightarrow B) \vee (B \Rightarrow C) &\Leftrightarrow & (\neg A \vee B) \vee (\neg B \vee C)\\
&\Leftrightarrow & \neg A \vee B \vee \neg B \vee C\\
&\Leftrightarrow & \neg A \vee (B \vee \neg B) \vee C\\
&\Leftrightarrow & \neg A \vee 1 \vee C\\
&\Leftrightarrow & 1.
\end{eqnarray*}

\end{answer}
\begin{answer}{1.20}
    Pokažimo $x + \frac{1}{x}\geq 2$. Ker $x\geq 0$, pomnožimo neenakost z $x$ in dobimo
$x^2 + 1 \geq 2x$ oziroma $(x- 1)^2\geq 0$. Slednje je očitno vedno res.
\end{answer}
\begin{answer}{1.21}
Recimo, da jih je končno mnogo $p_1,p_2,\ldots, p_n$. Potem  $p=p_1p_2\cdots p_n+1$ ni deljivo z nobenim pra"stevilom $p_i$ in $p_i\neq p$ za vsak $i$. Po definiciji je torej $p$ pra"stevilo, ki ni enako nobenemu prej"snjemu. Protislovje.
\end{answer}
\begin{answer}{1.22}
 Nasporotna trdtev je: obstaja naravno "stevilo, ki je večje od $1$.
\end{answer}
\begin{answer}{1.23}
Naj bo $x<2y$, to je, $2y-x>0$. Pokazali bomo: če je $3x> y$, potem je $7xy > 3x^2 + 2y^2$. Predpostavimo torej, da je $3x-y>0$. Potem je $(2y-x)(3x-y)= 7xy - 3x^2 - 2y^2>0$, to je, $7xy > 3x^2 + 2y^2$.
\end{answer}
\begin{answer}{1.24}

 ($\Rightarrow$) Predpostavimo, da sta različnih parnosti. Pi"simo  $m=2k$ in $n=2l+1$, vstavimo v izraz $m^2- n^2$ in rezultat sledi.

($\Leftarrow$) Pokažemo indirektno in sicer: Če sta $m$ in $n$ iste parnosti, potem je $m^2- n^2$ sodo. Obravnavaj oba primera.
\end{answer}
\begin{answer}{1.26}
    \begin{enumerate}[(i)]
        \item $(A\Rightarrow B) \wedge A \Rightarrow B$. Res je.
        \item $(A\Rightarrow B) \wedge \neg A \Rightarrow \neg B$. Ni nujno res.
    \end{enumerate}
\end{answer}
\begin{answer}{1.27}
\begin{itemize}
\item $((A\Rightarrow B) \wedge \neg A) \Rightarrow \neg B$. Ni nujno res.
    \item $((A\Rightarrow B) \wedge (B\Rightarrow C)) \Rightarrow (\neg C \Rightarrow \neg A)$. Res je.
\end{itemize}


\end{answer}
\begin{answer}{2.3}
\begin{eqnarray*}
x\in (A\cup C)\cap (B\setminus C) &\Leftrightarrow & (x\in A \vee x\in C) \wedge (x\in B \wedge x\notin C)\\
 &\Leftrightarrow & ((x\in A \vee x\in C) \wedge (x\notin C))\wedge x\notin B\\
&\Leftrightarrow & ((x\in A \wedge x\notin C) \vee (x\in C \wedge x\notin C)) \wedge
 x\in B\\
&\Leftrightarrow & x\in A \wedge x\notin C  \wedge x\in B\\
&\Leftrightarrow & x\in A \wedge x\in B  \wedge x\notin C \\
&\Leftrightarrow & x \in (A\cap B)\setminus C.
\end{eqnarray*}
\end{answer}
\begin{answer}{2.4}
    Direktno.
\end{answer}
\begin{answer}{2.6}
\begin{eqnarray*}
(x,y)\in A\times (B\cap C) &\Leftrightarrow & x \in A \wedge y\in B\cap C\\
&\Leftrightarrow & x \in A \wedge y\in B  \wedge y\in C\\
&\Leftrightarrow & x \in A \wedge x \in A\wedge y\in B  \wedge y\in C\\
&\Leftrightarrow & x \in A \wedge  y\in B  \wedge x \in A\wedge y\in C\\
&\Leftrightarrow & (x,y) \in A\times B \wedge  (x,y) \in A\times C\\
&\Leftrightarrow & (x,y) \in (A\times B)\cap   (A\times C).
\end{eqnarray*}
\end{answer}
\begin{answer}{2.9}
\begin{enumerate}
\item Napačna. Vzemi $A=\emptyset$, $B=\{\emptyset\}$, $B=\{\{\emptyset\}\}$.
\item Napačna. Vzemi isti primer kot v (a).
\item Pravilna. Dokaz s protislovjem. Recimo, da trditev ni pravilna. Naj bo $A\cap B\subseteq \overline{C}$, $A\cup C\subseteq B$  in naj obstaja $x\in A\cap C$. Torej je $x\in A$ in $x\in C$. Ker je po drugi predpostavki $A\cup C\subseteq B$, je $x\in B$. Sledi $x\in A \cap B$. Ker je po prvi predpostavki $A\cap B\subseteq \overline{C}$, je $x\in \overline{C}$. Protislovje, saj $x\in C$.
\item Napačna. Vzemi $A=C\neq B$.
\item Napačna. Vzemi tri paroma disjunktne neprazne množice.
\end{enumerate}
\end{answer}
\begin{answer}{2}
    Vse.
\end{answer}
\begin{answer}{2}
    Da: lahko vzamemo npr.~$A = \emptyset$ in poljubni različni množici $B\neq C$
(potem bo namreč $A\cap B = A\cap C = \emptyset$).

Ali pa: $A = \{1,2\}, B = \{2,3\}, C = \{2,4\}$.
\end{answer}
\begin{answer}{2}
\textbf{1.~ način}:

Očitno je
$(A\backslash B) \cap (B\backslash A)=\emptyset$.
Ob upoštevanju predpostavke $A\backslash B = B\backslash A$
zveza
$(A\backslash B) \cap (B\backslash A)=\emptyset$
postane
$A\backslash B =\emptyset$
in $B\backslash A =\emptyset$, kar pomeni
$A\subseteq B$ in $B\subseteq A$. Torej je $B = A$.\qed

\textbf{2.~ način}:

$A\backslash B = B\backslash A \sledi(A\cap B)\cup (A\backslash B) = (B\cap A)\cup (B\backslash A) \sledi A = B$.\qed
\end{answer}
\begin{answer}{2}
Ne drži. Že zato, ker množica na levi strani ni nujno kartezični produkt dveh množic.

Zgled: $A = \{1\}, B = C = \emptyset$.

\end{answer}
\begin{answer}{2.14}
    \begin{enumerate}
\item Pokažimo najprej inplikacijo v desno. Naj bo $A\subseteq B$. Pokažimo, da tedaj velja $A\cap \overline{B} = \emptyset$. Po predpostavki velja
\begin{equation}\label{eq:4}
(\forall x)(x\in A \Rightarrow x\in B).
\end{equation}
Recimo, da obstaja $x\in A\cap \overline{B}$. Potem
\begin{eqnarray*}
x\in A\cap \overline{B} &\Rightarrow & x\in A \wedge x\in \overline{B}\\
						&\Rightarrow & x\in A \wedge x \notin B\\
						&\Rightarrow & x\in B \wedge x \notin B \quad (\textrm{zaradi } (\ref{eq:4})),
\end{eqnarray*}
protislovje. Torej velja $A\cap \overline{B}=\emptyset$.

"Se obratna inkluzija. Naj bo $A\cap \overline{B}=\emptyset$. Pokažimo, da velja $A\subseteq B$. Vzemimo poljuben $x\in A$. Potem $x\notin \overline{B}$, saj je $A\cap \overline{B}=\emptyset$. Torej $x\in B$. Ker je bil $x$ poljuben, sledi $A\subseteq B$.


\item
\begin{eqnarray*}
A\setminus B & = & \{x\, ;\, x\in A \wedge x\notin B \}\\
             & = & \{x\, ;\, x\notin \overline{A} \wedge x  \in \overline{B} \}\\
             & = & \{x\, ; x\,  \in \overline{B} \wedge  x\notin \overline{A} \}\\
             & = & \overline{B} \setminus \overline{A}.
\end{eqnarray*}

\end{enumerate}

\end{answer}
\begin{answer}{3.6}
     ($\Rightarrow$) $x(R\circ S)y \Rightarrow y(R\circ S)x \Rightarrow (\exists z) ((y,z)\in S \wedge (z,x)\in R) \Rightarrow (z,y)\in S \wedge (x,z)\in R \Leftrightarrow (x,z)\in R \wedge (z,y)\in S \Rightarrow x(S\circ R)x$. Podobno $x(S\circ R) \Rightarrow x(R\circ S)y$.

$(\Leftarrow)$ $x(R\circ S)y \Leftrightarrow x(S\circ R)y \Leftrightarrow (\exists z)((x,z)\in R \wedge (z,y)\in S)\Rightarrow (\exists z)((z,x)\in R \wedge (y,z)\in S)\Rightarrow (y,x) \in R\circ S$.

\end{answer}
\begin{answer}{3.7}

(a)
$R\circ R = \{(a,e), (d,a), (e,c), (e,d)\}$.

Sledi
$(R\circ R)\cap S = \emptyset$. To pa je irefleksivna relacija.

(b) $S\circ R = \{(a,c), (e,c), (e,f)\}$

Ta relacija ni sovisna, saj $(a,b)\not\in S\circ R$
$(b,a)\not\in S\circ R$.

(c)
$S\circ S = \{(a,d), (f,c)\}$.

$S\cup (S\circ S)= \{(a, c), (a,d), (a, f), (d, c), (f, d), (f,c)\}$ .

Ta relacija je tranzitivna, saj velja $x(S\cup (S\circ S))y\inn
y(S\cup (S\circ S))z\sledi x(S\cup (S\circ S))y$.

(d)
$S^{-1} = \{(c, a), (f, a), (c, d), (d, f)\}$ .

$S^{-1}\cup R = \{(c, a), (f, a), (c, d), (d, f),
(a, c), (a, d), (d, e), (e, a)\}$ .

Relacija ni simetrična, saj je $(f,a)\in S^{-1}\cup R$ in $(a,f)\not\in S^{-1}\cup R$.
\end{answer}
\begin{answer}{3.37}
\begin{enumerate}
    \item Naj bo $\rel{R}$ delna urejenost na množici $A$. Po definiciji je $\rel{R}$ refleksivna, antisimetrična in tranzitivna. Definicija predurejenosti zahteva le, da je relacija refleksivna in tranzitivna. Ker $\rel{R}$ zadošča tema dvema lastnostma, je $\rel{R}$ predurejenost.
    \item Naj bo $A = \{a, b\}$ in naj bo $\rel{R} = \{(a, a), (b, b), (a, b), (b, a)\}$.
    \begin{itemize}
        \item \textbf{Refleksivnost:} $(a, a) \in \rel{R}$ in $(b, b) \in \rel{R}$, torej je refleksivna.
        \item \textbf{Tranzitivnost:} Edina para, ki bi lahko kršila tranzitivnost, sta $(a, b)$ in $(b, a)$, kar implicira $(a, a) \in \rel{R}$ (izpolnjeno); ter $(b, a)$ in $(a, b)$, kar implicira $(b, b) \in \rel{R}$ (izpolnjeno). Torej je tranzitivna.
        \item \textbf{Antisimetričnost:} $(a, b) \in \rel{R}$ in $(b, a) \in \rel{R}$, vendar $a \ne b$. Torej $\rel{R}$ NI antisimetrična.
    \end{itemize}
    Ker je $\rel{R}$ refleksivna in tranzitivna, je predurejenost. Ker ni antisimetrična, ni delna urejenost.
\end{enumerate}
\end{answer}
\begin{answer}{3.38}
\begin{enumerate}
    \item Standardna dodatna lastnost je \textbf{irefleksivnost} (ali protirefleksivnost).
    \item Dokaz:
    \begin{itemize}
        \item[($\Rightarrow$)] Naj bo $\rel{R}$ stroga delna urejenost. Po definiciji je stroga delna urejenost irefleksivna in tranzitivna relacija. (Lastnost antisimetričnosti je ob irefleksivnosti in tranzitivnosti odveč: če $(a, b) \in \rel{R}$ in $(b, a) \in \rel{R}$, potem zaradi tranzitivnosti sledi $(a, a) \in \rel{R}$. Toda ker je $\rel{R}$ irefleksivna, je to nemogoče. Torej ne moremo imeti hkrati $(a, b)$ in $(b, a)$, kar pomeni, da je antisimetričnost izpolnjena "na prazno".)
        \item[($\Leftarrow$)] Naj bo $\rel{R}$ tranzitivna in irefleksivna relacija. To je standardna definicija stroge delne urejenosti, torej je $\rel{R}$ stroga delna urejenost.
    \end{itemize}
    (Alternativno: če strogo delno urejenost definiramo kot irefleksivno, antisimetrično in tranzitivno relacijo, moramo le pokazati, da antisimetričnost sledi iz drugih dveh, kot je navedeno zgoraj).
\end{enumerate}
\end{answer}
\begin{answer}{3.39}
Naj bo $\rel{R}$ relacija na $A$.
\begin{itemize}
    \item \textbf{Iz linearne urejenosti $\rel{R}$ v strogo linearno urejenost $\rel{R}^-$:}
    Definirajmo $\rel{R}^- = \rel{R} \setminus \{(a, a) \mid a \in A\}$.
    Če je $\rel{R}$ linearna urejenost, je refleksivna, antisimetrična, tranzitivna in polna. $\rel{R}^-$ je po konstrukciji očitno irefleksivna. Je tranzitivna, ker je $\rel{R}$ tranzitivna, odstranitev refleksivnih parov pa ne ustvari novih kršitev tranzitivnosti. $\rel{R}^-$ je polna za različne elemente: ker je $\rel{R}$ polna, za $a \ne b$ velja $(a, b) \in \rel{R}$ ali $(b, a) \in \rel{R}$. Ker $a \ne b$, to niso refleksivni pari, zato $(a, b) \in \rel{R}^-$ ali $(b, a) \in \rel{R}^-$. Torej je $\rel{R}^-$ stroga linearna urejenost.

    \item \textbf{Iz stroge linearne urejenosti $\rel{S}$ v linearno urejenost $\rel{S}^+$:}
    Definirajmo $\rel{S}^+ = \rel{S} \cup \{(a, a) \mid a \in A\}$.
    Če je $\rel{S}$ stroga linearna urejenost, je irefleksivna, tranzitivna in polna za različne elemente. $\rel{S}^+$ je po konstrukciji očitno refleksivna. Je tranzitivna: vsaka kršitev tranzitivnosti bi morala vključevati vsaj en nov refleksivni par, recimo $(a, a) \in \rel{S}^+$ in $(a, b) \in \rel{S}^+$. To implicira $(a, b) \in \rel{S}$ in s tem $(a, b) \in \rel{S}^+$. Podobno za $(b, a) \in \rel{S}^+$ in $(a, a) \in \rel{S}^+$. Je antisimetrična: če $(a, b) \in \rel{S}^+$ in $(b, a) \in \rel{S}^+$, je to zaradi irefleksivnosti $\rel{S}$ možno le, če $a=b$. Je polna: za poljubna $a, b$, če $a=b$, velja $(a, a) \in \rel{S}^+$. Če $a \ne b$, potem $(a, b) \in \rel{S}$ ali $(b, a) \in \rel{S}$, kar pomeni $(a, b) \in \rel{S}^+$ ali $(b, a) \in \rel{S}^+$. Torej je $\rel{S}^+$ linearna urejenost.
\end{itemize}
Ker velja $(\rel{R}^-)^+ = \rel{R}$ in $(\rel{S}^+)^- = \rel{S}$, sta preslikavi inverzni, kar vzpostavi bijektivno korespondenco.
\end{answer}
\begin{answer}{3.5}

(a) $R = \{(5,2),(5,1),(4,2),(4,1),(5,3),(4,1),(2,1),(3,1)\}$.

(b) Noben element ni pod elementoma 4 in 5, torej sta 4 in 5 minimalna elementa.
Maksimalen element pa je samo eden: 1.

(c) Množica $S$ nima prvega elementa. Elementa 4 in 5 sta sicer minimalna, vendar nobeden
od njiju ni pod drugim. Množica $S$ pa ima zadnji element: to je element 1, saj so vsi drugi elementi
pod njim.

(d) Ne, saj ni sovisna: Elementa 2 in 3 nista primerljiva: $(2,3)\not\in R$, $(3,2)\not\in R$.\qed
\end{answer}
