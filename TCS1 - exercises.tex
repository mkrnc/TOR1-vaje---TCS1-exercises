\RequirePackage{silence} % :-\
    \WarningFilter{scrbook}{Usage of package `titlesec'}
    \WarningFilter{titlesec}{Non standard sectioning command detected}
\documentclass[11pt,paper=b5,footinclude,headinclude]{scrbook} % KOMA-Script book
\usepackage[T1]{fontenc}
\usepackage[style=arsclassica, parts=false, eulermath=false, palatino=true, eulerchapternumbers=true]{classicthesis}


%\documentclass[11pt,a5paper,footinclude,headinclude]{scrbook} % KOMA-Script book

\usepackage{xspace,xcolor,afterpage}
\usepackage{url}
\usepackage{enumitem}
\usepackage{comment}
\usepackage{graphicx}
\usepackage{amsfonts,amsmath,amssymb,amsthm}
\usepackage{graphics}
\usepackage{color}
\usepackage{fullpage}
\usepackage{makecell}
%\usepackage{times}
%\usepackage{txfonts}
\def\P {{\cal P}}
\def\ali {{~\vee~}}
\def\inn {{~\wedge~}}
\def\sledi {{~\Rightarrow~}}
\def\brez {{\,\setminus\,}}
\def\cee {{~\Leftrightarrow~}}
\def\zgled{\paragraph{Example:}}
\def\kz{{\hfill{\S}}}% konec zgleda
\newenvironment{example}{\paragraph{Example:}}{\hfill \S}

\theoremstyle{remark}
\newtheorem*{remark}{Remark}
\newtheorem*{lemma}{Lemma}
\newtheorem*{corollary}{Corollary}
\theoremstyle{definition} %theorem
\newtheorem*{definition}{Definition}
\theoremstyle{theorem} %theorem
\newtheorem*{theorem}{Theorem}
\newtheorem*{proposition}{Proposition}

%vars
\newcommand{\myTitle}{Theoretical Computer Science\xspace}
\newcommand{\mySubtitle}{Discrete Structures for Computer Science Students \xspace} 				%dodaj podnaslov if needed
\newcommand{\myName}{UP FAMNIT \xspace}
\newcommand{\myPublisher}{Matjaž Krnc}
\newcommand{\myMonth}{Fall}
\newcommand{\myYear}{2024}
\newcommand{\verzija}{Version 0.1\xspace}
\newcommand{\shortAuthors}{M. Krnc}
\newcommand{\myAuthors}{Matjaž Krnc}
\newcommand{\myISBN}{978-961-XXX-XXX-X }
\newcommand{\myRepo}{\url{https://github.com/mkrnc/TOR1-vaje---TCS1-exercises.git}}


\begin{document}
\input{Titlepage}
\input{en/cip}
\newpage
\section*{Preface}
   

Those notes are supposed to be parsed together with explanations from the lectures.
Any questions or found errors should be 
addressed to \url{matjaz.krnc@upr.si}, or 
raised as an issue in our public repository 
\begin{center}
    \myRepo.    
\end{center}


Among most notable student contributors are:


\input{en/kolofon}

\tableofcontents


\chapter{Mathematical Logic}


\section{Basic exercises}

\begin{enumerate}[label=\textbf{Problem \arabic*.}]
    \item 
The following two propositions are given:
A: "It is cold outside."
B: "It is raining outside."

Write the following compound propositions in natural language:
\begin{itemize}
    \item (a) $\neg A$
    \item (b) $A \land B$
    \item (c) $A \lor B$
    \item (d) $B \lor \neg A$
\end{itemize}

\item 
The following two propositions are given:
A: "Janez is rich."
B: "Janez is happy."

Write the following propositions symbolically:
\begin{itemize}
    \item (a) If Janez is rich, then he is unhappy.
    \item (b) Janez is neither happy nor rich.
    \item (c) Janez is happy only if he is poor.
    \item (d) Janez is poor if and only if he is unhappy.
\end{itemize}

\item 
Find the truth tables for the examples from the previous task.


\item
Is the following reasoning correct?

\begin{itemize}
    \item Premise 1: "I think, therefore I am."
    \item Premise 2: "I think, therefore I reason."
    \item Conclusion: "I am, therefore I reason."
\end{itemize}

\textbf{Solution:} The logical implication is:

\[
(A_1 \Rightarrow A_2) \land (A_1 \Rightarrow A_3) \Rightarrow (A_2 \Rightarrow A_3)
\]

For \( A_1(d) = 0 \), \( A_2(d) = 1 \), \( A_3(d) = 0 \), this implication is false. Therefore, the reasoning is incorrect.



\item
The following two propositions are given:
$A$: ``Andrej speaks French.'' and $B$: ``Andrej speaks Danish.''
Write the following compound propositions in natural language:

(a) $A\ali B$

(b) $A\inn B$

(c) $A\inn \neg B$

(d) $\neg A\ali \neg B$

(e) $\neg \neg A$

(f) $\neg (\neg A\inn \neg B)$


\item \label{ex:newspapers} Given the propositions:\\
$A:$ ``John reads The New York Times.''\\
$B:$ ``John reads The Wall Street Journal.''\\
$C:$ ``John reads The Daily Mail.''\\
\\
Transcribe the following statements into symbolic propositions:
\begin{enumerate}
\item John reads The New York Times, but not The Wall Street Journal.
\item Either John reads both The New York Times and The Wall Street Journal,
or he does not read The New York Times and The Wall Street Journal.
\item It is not true that John reads The New York Times, and does not read
The Daily Mail.
\item It is not true that John reads The Daily Mail or The Wall Street Journal,
and not The New York Times.
\end{enumerate}
\item Find the truth tables for the symbolic propositions from Exercise \ref{ex:newspapers}.


\item For three lines $p,q,r$ we may construct also geometric propositions.
Suppose that the following is true:
\[
(p||q)\wedge(p\cap q\neq\emptyset)\wedge(q\cap r\neq\emptyset).
\]
What can you say about the lines $p,q,r$?



\item Express the propositions below with connectives $\wedge$ and $\neg$
only!
\begin{enumerate}
\item $A\vee B$
\item $A\Rightarrow B$
\item $A\Leftrightarrow B$
\end{enumerate}

\end{enumerate}

\section{Knights and servants (kneves)}
Knights always tell the truth, while servants always lie.


\begin{enumerate}[resume, label=\textbf{Problem \arabic*.}]

\item 
 Artur: "It is not true that Cene is a servant."
 
Bine: "Cene is a knight or I am a knight."

Cene: "Bine is a servant."

For each of them, determine whether they are knights or servants!

% \textbf{Solut6 0$ and  $B = 1$. 

\item Artur: "Cene is a servant or Bine is a servant."

Bine: "Cene is a knight and Artur is a knight."

For each of them, determine whether they are knights or servants!

% \textbf{Solution:}
% Let us denote: $A$ – Arthur is a knight, $B$ – Bine is a knight, $C$ – Cene is a knight.

% The following compound statement holds:

% \[
% (A \iff \neg C \lor \neg B) \land (B \iff C \land A)
% \]

% With the help of the truth table, we see that the statement is true only for the set $A = 1$ and $B = C = 0$.

    \item Let us analyze the statements made by A, B, C, and D:

\begin{itemize}
    \item A: "D is a servant and C is a servant."
    \item B: "If A and D are servants, then C is a servant."
    \item C: "If B is a servant, then A is a knight."
    \item D: "If E is a servant, then both C and B are servants."
\end{itemize}

%%solution
% \textbf{Solution:} Let \( A \) represent "A is a knight", etc. We seek the only solution \( d \) such that the following is true:

% \[
% A_1 \land B_1 \land C_1 \land D_1
% \]

% where:

% \[
% A_1 : A \iff (\neg D \land \neg C)
% \]
% \[
% B_1 : B \iff (\neg A \land \neg D \Rightarrow \neg C)
% \]
% \[
% C_1 : C \iff (\neg B \Rightarrow A)
% \]
% \[
% D_1 : D \iff (\neg E \Rightarrow \neg C \land \neg B)
% \]

% Since the truth table would contain 32 rows, we solve this by analyzing cases.

% \subsection*{Case 1: \( A(d) = 1 \)}

% Given \( A_1 \), \( D(d) = 0 \) and \( C(d) = 0 \). Substituting into \( C_1 \) with \( A(d) = 1 \) and \( C(d) = 0 \), we get:

% \[
% \neg (\neg B \Rightarrow 1) \Rightarrow \neg(B \lor 1) \Rightarrow \neg 1
% \]

% which is a false statement. Hence, this case is not possible.

% \subsection*{Case 2: \( A(d) = 0 \)}

% Given \( A_1 \), either \( C(d) = 1 \) or \( D(d) = 1 \).

% \subsection*{Case 2.1: \( C(d) = 1 \)}

% Since \( C_1 \), \( \neg B \Rightarrow 0 \), implies \( \neg B = 0 \), hence \( B(d) = 1 \).

% Substituting into \( B_1 \) with \( A(d) = 0 \), \( B(d) = 1 \), and \( C(d) = 1 \), we get:

% \[
% 1 \land \neg D \Rightarrow 0 \Rightarrow \neg D = 0
% \]

% thus, \( D(d) = 1 \).

% Substituting into \( D_1 \), we get:

% \[
% \neg E \Rightarrow 0 \land 0
% \]

% Hence \( E(d) = 1 \).

% \subsection*{Case 2.2: \( C(d) = 0 \) and \( D(d) = 1 \)}

% From \( B_1 \), we get \( B(d) = 1 \), but the statement \( C_1 \) becomes false: \( 0 \iff (0 \Rightarrow 1) \).

% Thus, \( B \), \( C \), \( D \), and \( E \) are knights, while \( A \) is a servant.



\item Solve the following exercises about knights and servants:
\begin{itemize}
  \item Arthur: ``It is not true that Bine is a servant."
  \item Bine: ``We are not both of the same kind.''
\end{itemize}




\item 
Now Arthur and Bine say the following:
\begin{itemize}
 \item Arthur: ``Me and Bine are not of the same kind.''
 \item Bine: ``Exactly one of us is a knight.''
\end{itemize}


\item Knights and servants! 
\begin{enumerate}

\item Arthur: Chloe or Bob are servants.
\item 
Bob: Cene and Arthur are knights.
\end{enumerate}


\end{enumerate}



\section{Canonical Forms}
\begin{enumerate}[resume, label=\textbf{Problem \arabic*.}]
    \item Find the canonical disjunctive normal form (DNF) and the canonical conjunctive normal form (CNF) for the following propositions:
    \begin{itemize}
        \item[(i)] $\neg(A \land B) \Rightarrow (\neg B \Rightarrow A)$
        \item[(ii)] $\neg(A \lor B) \land (A \Rightarrow B)$
    \end{itemize}




\item For the following compound proposition find  a truth table, determine DNF, CNF and draw the corresponding circuit.
$$
(A \Rightarrow (B\Rightarrow C)) \Rightarrow ((A\Rightarrow B)\Rightarrow (A \Rightarrow C)).
$$

\end{enumerate}
\section{Switching circuits}
\begin{enumerate}[resume, label=\textbf{Problem \arabic*.}]
    \item For the circuits in Figure \ref{fig:circuits}, find the corresponding compound propositions.\label{ex:circuits}
\end{enumerate}

\begin{figure}
    \centering
    \includegraphics[width=0.6\linewidth]{img/vez1.pdf}
    \includegraphics[width=0.7\linewidth]{img/vez2.pdf}
    \includegraphics[width=0.8\linewidth]{img/vez3.pdf}
    \caption{Switching circuits for \ref{ex:circuits}}
    \label{fig:circuits}
\end{figure}



\begin{enumerate}[resume, label=\textbf{Problem \arabic*.}]
    \item For the following compound proposition, find a truth table, determine DNF, CNF, and draw the corresponding circuit.
\[
(A \Rightarrow (B \Rightarrow C)) \Rightarrow ((A \Rightarrow B) \Rightarrow (A \Rightarrow C))
\]
    \item Find a compound proposition $I$ such that
\[
(A \Rightarrow (I \Rightarrow \neg B)) \Rightarrow (A \land B) \lor I
\]
is a tautology.
\end{enumerate}


\section{Logical Implications}

\begin{enumerate}[resume, label=\textbf{Problem \arabic*.}]
    \item Prove the following logical equivalences:
        \begin{itemize}
            \item[(1)] $(A \Rightarrow B) \land (B \Rightarrow C) \Rightarrow (A \Rightarrow C)$
            \item[(2)] $(A \Rightarrow B) \Rightarrow ((C \Rightarrow A) \Rightarrow (C \Rightarrow B))$
            \item[(3)] $(A \Rightarrow B) \Rightarrow ((B \Rightarrow C) \Rightarrow (A \Rightarrow C))$
            \item[(4)] $(A \Rightarrow B) \Rightarrow (A \land C \Rightarrow B \land C)$
            \item[(5)] $(A \Rightarrow B) \Rightarrow (A \lor C \Rightarrow B \lor C)$
            \item[(6)] $(A \Leftrightarrow B) \land (B \Leftrightarrow C) \Rightarrow (A \Leftrightarrow C)$
            \item[(7)] $(A \Leftrightarrow B) \Rightarrow (A \Rightarrow B)$
            \item[(8)] $(A \Leftrightarrow B) \Rightarrow (B \Rightarrow A)$
            \item[(9)] $A \land (A \Leftrightarrow B) \Rightarrow B$
            \item[(10)] $\neg A \land (A \Leftrightarrow B) \Rightarrow \neg B$
            \item[(11)] $B \Rightarrow (A \Leftrightarrow A \land B)$
            \item[(12)] $\neg B \Rightarrow (A \Leftrightarrow A \lor B)$
            \item[(13)] $(A \Rightarrow (B \land \neg B)) \Rightarrow \neg A$
        \end{itemize}
    \item Simplify the following logical proposition:
\[
(A \Rightarrow B) \lor (B \Rightarrow C)
\]

% \emph{Solution.} 
% \begin{eqnarray*}
% (A\Rightarrow B) \vee (B \Rightarrow C) &\Leftrightarrow & (\neg A \vee B) \vee (\neg B \vee C)\\
% &\Leftrightarrow & \neg A \vee B \vee \neg B \vee C\\
% &\Leftrightarrow & \neg A \vee (B \vee \neg B) \vee C\\
% &\Leftrightarrow & \neg A \vee 1 \vee C\\
% &\Leftrightarrow & 1.
% \end{eqnarray*}
\end{enumerate}

\section{Proofs}
\begin{enumerate}[resume, label=\textbf{Problem \arabic*.}]
    \item Show that the following propositions are logical implications (i.e., tautologies where the main connective is implication):
        \begin{itemize}
            \item[(i)] $A \land (A \Rightarrow B) \Rightarrow B$
            \item[(ii)] $\neg B \land (A \Rightarrow B) \Rightarrow \neg A$
            \item[(iii)] $\neg A \land (A \lor B) \Rightarrow B$
            \item[(iv)] $(A \Rightarrow B) \land (B \Rightarrow C) \Rightarrow (A \Rightarrow C)$
            \item[(v)] $A \land (A \Leftrightarrow B) \Rightarrow B$
        \end{itemize}
    \item Are the following propositions logical implications?
\begin{itemize}
    \item[(i)] $(A \Rightarrow B) \land (A \Rightarrow C) \land A \Rightarrow B \land C$
    \item[(ii)] $\neg(A \lor B) \land (A \lor C) \land (D \Rightarrow C) \Rightarrow D$
    \item[(iii)] $(A \Rightarrow B) \land (A \Rightarrow C) \land (D \land E \Rightarrow F) \land (C \Rightarrow E) \Rightarrow F$
\end{itemize}



\item Show that the following propositions are logical implications (a tautology where the main connective is implication).
\begin{enumerate}
\item[(i)] $A \wedge (A \Rightarrow B) \Rightarrow B$
\item[(ii)] $\neg B \wedge (A \Rightarrow B) \Rightarrow \neg A$
\item[(iii)] $\neg A \wedge (A \vee B) \Rightarrow B$
\item[(iv)] $(A \Rightarrow B) \wedge (B \Rightarrow C) \Rightarrow (A \Rightarrow C)$
\item[(v)] $A \wedge (A \Leftrightarrow B) \Rightarrow B$
\end{enumerate}

% \emph{Rešitev.} (i) Recimo $A \wedge (A \Rightarrow B)$ pravilna, $B$ pa nepravilna. Potem je $A$ pravilna in $A\Rightarrow B$ pravilna. Sledi $B$ pravilna. Protislovje. 


\item Are the following propositions logical implications?
\begin{enumerate}
\item[(i)] $(A \Rightarrow B ) \wedge (A \Rightarrow C) \wedge A \Rightarrow B \wedge C$
\item[(ii)] $\neg (A \vee B) \wedge (A\vee C) \wedge (D\Rightarrow C) \Rightarrow D$
\item[(iii)] $(A\Rightarrow B) \wedge (A\Rightarrow C) \wedge (D\wedge E \Rightarrow F) \wedge (C\Rightarrow E) \Rightarrow F$
\end{enumerate}

\item With a direct proof show:
\begin{quote}
    If $n$ is even, then so is $n^2 +3n$.
\end{quote}
Is the converse also true?






    \item Use a direct proof of the implication to show: If a real number $x$ is non-negative, then the sum of the number $x$ and its reciprocal is greater than or equal to $2$.

    % \emph{Solution.} We need to show that $x + \frac{1}{x} \geq 2$. Since $x \geq 0$, we can multiply the inequality by $x$ to get
    % \[
    % x^2 + 1 \geq 2x
    % \]
    % or equivalently,
    % \[
    % (x - 1)^2 \geq 0.
    % \]
    % This is obviously always true.

    \item Use contradiction to show that there are infinitely many prime numbers.

    % \emph{Solution.} Suppose there are only finitely many primes $p_1, p_2, \ldots, p_n$. Then the number $p = p_1 p_2 \cdots p_n + 1$ is not divisible by any prime $p_i$ and $p_i \neq p$ for each $i$. By definition, $p$ is therefore a prime number that is different from each of the previous ones. This is a contradiction.

    \item Find the error in the following proof.

    \textbf{Statement:} $1$ is the largest natural number.

    \textbf{Proof} (by contradiction):
    Suppose the opposite. Let $n > 1$ be the largest natural number. Since $n$ is positive, we can multiply the inequality $n > 1$ by $n$, giving
    \[
    n > 1 \Leftrightarrow n^2 > n.
    \]
    We have found that $n^2$ is greater than $n$, which contradicts the assumption that $n$ is the largest natural number. Therefore, the assumption was incorrect, and $1$ is the largest natural number.

    % \emph{Solution.} The opposite statement is: there exists a natural number greater than $1$.


\item Let  $x$ and $y$ be real numbers such that $x<2y$. 
By an indirect proof show: 
\begin{quote}
    If $7xy\leq 3x^2 + 2y^2$, then $3x\leq y$.
\end{quote}

% \emph{ Solution (in slovene).} Naj bo $x<2y$, to je, $2y-x>0$. Pokazali bomo: če je $3x> y$, potem je $7xy > 3x^2 + 2y^2$. Predpostavimo torej, da je $3x-y>0$. Potem je $(2y-x)(3x-y)= 7xy - 3x^2 - 2y^2>0$, to je, $7xy > 3x^2 + 2y^2$.

    \item Prove the following equivalence in two parts: Let $m$ and $n$ be integers. Then $m$ and $n$ have different parities if and only if $m^2 - n^2$ is odd.

    % \emph{Solution.} ($\Rightarrow$) Assume that $m$ and $n$ have different parities. Write $m = 2k$ and $n = 2l + 1$, substitute into the expression $m^2 - n^2$, and the result follows.

    % ($\Leftarrow$) We show indirectly: If $m$ and $n$ have the same parity, then $m^2 - n^2$ is even. Consider both cases.

    \item Using an "if and only if" proof, show that $ac \mid bc \Leftrightarrow a \mid b$.

    \item Is the following inference correct?
    \begin{enumerate}[label=(\roman*)]
        \item If today is Wednesday, I will have a tutorial. Today is Wednesday. Conclusion: I will have a tutorial.

        % \emph{Solution.} $(A \Rightarrow B) \wedge A \Rightarrow B$. This is true.

        \item If I study, I will pass the exam. I did not study. Conclusion: I will not pass the exam.

        % \emph{Solution.} $(A \Rightarrow B) \wedge \neg A \Rightarrow \neg B$. This is not necessarily true.

        \item A student took the city bus to the exam. He thought, "If the next traffic light is green, I will pass the exam." When the bus reached the next light, it was not green, so the student said to himself, "Darn, I'll fail again."

        % \emph{Solution.} $((A \Rightarrow B) \wedge \neg A) \Rightarrow \neg B$. This is not necessarily true.

        \item An engineer who understands theory always designs a good circuit. A good circuit is economical. Therefore, an engineer who designs an uneconomical circuit does not understand theory.

        % \emph{Solution.} $((A \Rightarrow B) \wedge (B \Rightarrow C)) \Rightarrow (\neg C \Rightarrow \neg A)$. This is true.
    \end{enumerate}



\item Which of the following propositions are correct where the language of the conversation are real numbers?
\begin{enumerate}
\item[(i)] $(\forall x)(\exists y)(x+y=0)$.
\item[(ii)] $(\exists x)(\forall y)(x+y=0)$.
\item[(iii)] $(\exists x)(\exists y)(x^2+y^2 =-1)$.
\item[(iv)] $(\forall x)[x>0 \Rightarrow (\exists y)(y<0 \wedge xy>0)]$.
\end{enumerate} 
\end{enumerate} 

\section{Proofs by Induction}

% Let $P(n)$ be a statement which depends on an element from  well-ordered set, say $n = 1, 2, 3, \ldots$. Then $P(n)$ is true for all $n$ if:
% \begin{itemize}
%     \item $P(1)$ is true (the base case).
%     \item Prove that $P(k)$ is true implies that $P(k + 1)$ is true. This is sometimes broken into two steps, but they go together: Assume that $P(k)$ is true, then show that with this assumption, $P(k + 1)$ must be true.
% \end{itemize}

% \section*{Exercises}
\begin{enumerate}[resume, label=\textbf{Problem \arabic*.}]
    \item Prove each using induction:
    \begin{enumerate}
        \item[(a)] $\sum_{i=1}^n i = \frac{n(n + 1)}{2}$
        \item[(b)] $\sum_{i=1}^n i^2 = \frac{n(n + 1)(2n + 1)}{6}$
        \item[(c)] $\sum_{i=1}^n 2^{i-1} = 2^n - 1$
        \item[(d)] $\sum_{i=1}^n i^3 = \frac{n^2(n + 1)^2}{4}$
        \item[(e)] $\frac{1}{1 \cdot 2} + \frac{1}{2 \cdot 3} + \cdots + \frac{1}{n(n + 1)} = \frac{n}{n + 1}$
        \item[(f)] $\sum_{i=1}^n \frac{1}{i(i + 1)} = \frac{n}{n + 1}$
        \item[(g)] $\sum_{i=1}^n (2i - 1) = n^2$
        \item[(h)] $n! > 2^n$ for $n \geq 4$.
        \item[(i)] $2^{n+1} > n^2$ for all positive integers.
    \end{enumerate}
    \item This exercise refers to the Fibonacci sequence:

\[
1, 1, 2, 3, 5, 8, 13, 21, 34, \ldots
\]

The sequence is defined recursively by $f_1 = 1$, $f_2 = 1$, then $f_{n+1} = f_n + f_{n-1}$ for each $n > 2$. As before, prove each of the following using induction. You might investigate each with several examples before you start.
    \begin{enumerate}
        \item[(a)] $f_1 + f_2 + \cdots + f_n = f_{n+2} - 1$
        \item[(b)] $f_1^2 + f_2^2 + \cdots + f_n^2 = f_n f_{n+1}$
        \item[(c)] $f_1 + f_3 + f_5 + \cdots + f_{2n-1} = f_{2n}$
    \end{enumerate}
\end{enumerate}



\chapter{Set theory}
\begin{enumerate}

\item Let $A = \{ x \in \mathbb{N}; x < 7\}, B = \{x \in  \mathbb{Z}; |x - 2| < 4\}$ and $C = \{x \in\mathbb{R}; x^3 -  4x = 0\}$.
\begin{enumerate}
\item[(i)]  Write down the elements for all three sets.
\item[(ii)] Find $A \cup C, B \cap C, B \setminus C, (A \setminus B) \setminus C$ and $A \setminus (B \setminus C)$.
\end{enumerate}

\item Let  $\mathbb{Z}$ be a universal set and let  $P$ denote the set of all prime numbers, and $S$ the set of all even integers. Write the following propositions in terms of set theory:
\begin{itemize}
\item[(i)] There exists an even prime number. \quad [$P\cap S \neq \emptyset$]
\item[(ii)] $0$ is an integer, but it is not natural number. \quad [$0 \in \mathbb{Z}\setminus \mathbb{N}$]
\item[(iii)] Every natural number is an integer. \quad [$\mathbb{N}\subseteq \mathbb{Z}$]
\item[(iv)] Not every integer is a natural number. \quad [$\mathbb{Z}\nsubseteq \mathbb{N}$]
\item[(v)] Every prime number except 2 is odd. \quad [$P\setminus \{2\} \subseteq \overline{S}$]
\item[(vi)] 2 is an even prime number. \quad [$2\in S\cap P$]
\end{itemize}

\item Let  $A, B, C$ and $D$  be subsets of some universal set  $U$. Simplify the following expression
$$\overline{(\overline{(A\cup B)} \cap \overline{(\overline{A} \cup C)})}\setminus \overline{D}.$$

\item Show that $(A\cup C)\cap (B\setminus C) = (A\cap B)\setminus C$.

\emph{ Rešitev.} 
\begin{eqnarray*}
x\in (A\cup C)\cap (B\setminus C) &\Leftrightarrow & (x\in A \vee x\in C) \wedge (x\in B \wedge x\notin C)\\
 &\Leftrightarrow & ((x\in A \vee x\in C) \wedge (x\notin C))\wedge x\notin B\\
&\Leftrightarrow & ((x\in A \wedge x\notin C) \vee (x\in C \wedge x\notin C)) \wedge
 x\in B\\
&\Leftrightarrow & x\in A \wedge x\notin C  \wedge x\in B\\
&\Leftrightarrow & x\in A \wedge x\in B  \wedge x\notin C \\
&\Leftrightarrow & x \in (A\cap B)\setminus C. 
\end{eqnarray*}

\item (Zadnja lastnost pri uniji) Prove that $A\subseteq C  \wedge B\subseteq C \Rightarrow A\cup B\subseteq C$.

\emph{ Rešitev.} Direktno.

\item (Predzadnja lastnost pri preseku) Prove that $A\subseteq  B \Leftrightarrow A\cap B = A$.

\emph{ Rešitev.} V dveh delih.

\item (Predzadnja lastnost pri kartezičnemu produktu) Prove that $A\times (B\cap C) = (A\times B)\cap (A\times C)$.

\emph{ Rešitev.} 
\begin{eqnarray*}
(x,y)\in A\times (B\cap C) &\Leftrightarrow & x \in A \wedge y\in B\cap C\\
&\Leftrightarrow & x \in A \wedge y\in B  \wedge y\in C\\
&\Leftrightarrow & x \in A \wedge x \in A\wedge y\in B  \wedge y\in C\\
&\Leftrightarrow & x \in A \wedge  y\in B  \wedge x \in A\wedge y\in C\\
&\Leftrightarrow & (x,y) \in A\times B \wedge  (x,y) \in A\times C\\
&\Leftrightarrow & (x,y) \in (A\times B)\cap   (A\times C).
\end{eqnarray*}

\item (Predzadnja lastnost pri razliki) Prove that $(A\cap B )\setminus B = \emptyset$.
\emph{ Rešitev.} 
\begin{eqnarray*}
x\in (A\cap B )\setminus B  &\Leftrightarrow & x \in (A\cap B)  \wedge x\notin B\\
&\Leftrightarrow & (x\in A\wedge x\in  B ) \wedge x\notin B\\
&\Leftrightarrow & x\in A\wedge (x\in  B  \wedge x\notin B)\\
&\Leftrightarrow & x\in \emptyset.
\end{eqnarray*}

\item Determine the following sets:
\begin{enumerate}
\item[(i)] $\{\emptyset, \{\emptyset\}\}\setminus \emptyset$ \quad [$\{\emptyset, \{\emptyset\}\}$]
\item[(ii)] $\{\emptyset, \{\emptyset\}\}\setminus \{\emptyset\}$
\item[(iii)] $\{\emptyset, \{\emptyset\}\}\setminus \{\}\emptyset\}\}$
\item[(iv)] $\{1,2,3,\{1\}, \{5\}  \}\setminus \{2,\{3\},5\}$
\end{enumerate}

\item Which of the following propositions are correct for arbitrary sets $A, B$ and $C$:
\begin{enumerate}
\item If $A\in B$ and $B\in C$, then $A\in C$.
\item If $A\subseteq B$ and $B\in C$, then $A\in C$.
\item If $A\cap B\subseteq \overline{C}$ and $A\cup C \subseteq B$, then $A\cap C = \emptyset$.
\item If $A\neq B$ and $B\neq C$, then $A\neq C$.
\item If $A\subseteq \overline{(B\cup C)}$ and $B\subseteq \overline{(A\cup C)}$, then $B=\emptyset$.
\end{enumerate}

\emph{ Rešitev.}
\begin{enumerate}
\item Napačna. Vzemi $A=\emptyset$, $B=\{\emptyset\}$, $B=\{\{\emptyset\}\}$.
\item Napačna. Vzemi isti primer kot v (a).
\item Pravilna. Dokaz s protislovjem. Recimo, da trditev ni pravilna. Naj bo $A\cap B\subseteq \overline{C}$, $A\cup C\subseteq B$  in naj obstaja $x\in A\cap C$. Torej je $x\in A$ in $x\in C$. Ker je po drugi predpostavki $A\cup C\subseteq B$, je $x\in B$. Sledi $x\in A \cap B$. Ker je po prvi predpostavki $A\cap B\subseteq \overline{C}$, je $x\in \overline{C}$. Protislovje, saj $x\in C$. 
\item Napačna. Vzemi $A=C\neq B$.
\item Napačna. Vzemi tri paroma disjunktne neprazne množice.
\end{enumerate}

\item Find $\mathcal{P}(A)$, where $A=\{a,b,c,d\}$.

\item Let $A=\{\{1,2,3\}, \{4,5\}, \{6,7,8\}\}$.
\begin{enumerate}
\item[(i)] Write down the elements of  $A$.
\item[(ii)] Is it true?\\
 (a) $1\in A$ \quad (b) $\{1,2,3\}\subseteq A$ \quad (c)  $\{6,7,8\}\in  A$ \quad  (d)  $\{\{4,5\}\}\subseteq A$\\
  (e) $\emptyset\in A$ \quad(f) $\emptyset\subseteq A$
\end{enumerate}


\item  Show that $A\times (B\cap C) = (A\times B)\cap (A\times C)$.



\item Let $A, B$ in $C$ be arbitrary subsets of the universal set $U= A \cup B \cup C$. Show the following propositions:
\begin{enumerate}
\item  $A\setminus B \subseteq \overline{B}$.
\item  $(A\setminus B)\cap B = \emptyset $.
\item  $A\cap B \subseteq C \Leftrightarrow A\subseteq \overline{B} \cup C$.
\item  $(A\setminus B) \cup B = A \Leftrightarrow B \subseteq A$.
\item  If $B\subseteq A$, then$B\times B = (B\times A)\cap (A\times B)$.
\item Let $A$ be a nonempty set. Which of the following sets 
$$\emptyset,\{\emptyset\}, A, \{A\}, \{A,\emptyset\}$$
are elements and which are subsets of (i) $\mathcal{P}(A)$ and (ii) $\mathcal{P}(\mathcal{P}(A))$?
\item  Is it true that  $\mathcal{P}(A\times B) = \mathcal{P}(A) \times \mathcal{P}(B)$?
\end{enumerate}


\end{enumerate}



\chapter{Relations}

\begin{enumerate}


\item Let $S=\{1,2,3,4,5\}$. 
\begin{enumerate}
    \item Is $R=\{(1,2),(2,3), (3,5), (2,4), (5,1)\}$ a binary relation?
    \item Find the domain $\mathcal{D} R$ and the range $\mathcal{Z} R$ of $R$.
    \item 
  Determine the inverse relation $R^{-1}$ and  $\mathcal{D} R^{-1}$ and  $\mathcal{Z} R^{-1}$.
\end{enumerate}

\item Let $R=\{(1,1),(2,1), (3,3), (1,5)\}$  and $T=\{(1,4),(2,1), (2,2), (2,5)\}$ be binary relations. \begin{enumerate}
    \item 
Determine the compositions $R\circ T$ and $T\circ R$. 
\item Is it true that $R\circ T = T \circ R$?
\end{enumerate}

\item Let  $S=\{1,2,3,4,5,6,7\}$. Define
$$R= \{(x,y)\,|\, x-y \textrm{ is divisible by  }  3\} \quad \mathrm{ in } \quad  T= \{(x,y)\,|\, x-y \geq 3\}.$$
Determine $R,T, R\circ R$.


\item Let  $S= \mathbb{R}$. On $S$ we define the relation $R$ as follows
$$(\forall x)(\forall y)(x R y \Leftrightarrow y \geq x +3).$$
Is $R$ reflexive, symmetric, transitive or strict total?

\item Let  $S=\{1,2,3,4\}$. We have the following relations
\begin{enumerate}
\item[(i)] $R_1= \{(1,1),(1,2),(2,3), (1,3), (4,4)\}$,
\item[(ii)] $R_2= \{(1,1),(1,2),(2,1), (2,2), (3,3), (4,4)\}$,
\item[(iii)] $R_3= \{(1,3),(2,1)\}$,
\item[(iv)] $R_4= \emptyset$,
\item[(v)] $R_5= S\times S$.
\end{enumerate}
Which of the following properties hold for each relation: reflexive, symmetric, antisymmetric, transitive?

\item Let $R$ and $S$ be symmetric relations. Show: $R\circ S$ symmetric $\Leftrightarrow R\circ S = S \circ R$.

\item Let $S= \{m\in \mathbb{N}\,|\, 1\leq n \leq 10\}$ in $R=\{(m,n)\in S\times S\,|\, 3|m-n\}$.
Is $R$ an equivalence relation? If yes, determine the corresponding equivalence classes and the factor set.

\section{Equivalences}
\item Let $S = \mathbb{Z}\times \mathbb{Z}$ and define the relation $R$ as follows
$$(a,b)R(c,d)\Leftrightarrow ad = bc.$$
Show that $R$ is an equivalence relation  and find the corresponding equivalence classes.

\item Let  $S =  \mathbb{R}^2$ and define the relation $R$ as follows
$$(x_1,y_1)R(x_2,y_2)\Leftrightarrow x_1^2 + y_1^2 = x_2^2 + y_2^2.$$
Show that $R$ is an equivalence relation  and find the equivalenec class $R[(7,1)]$.



\end{enumerate}
\section{Functions}
\begin{enumerate}

\item Let $A = \{1,2,3,4\}$, $B = \{x,y,z\}$, $C = \{a,b\}$. You are given functions $f:A\to B$ and $g:B\to C$.

\[ f = \{(1,x),(2,y),(3,y),(4,x)\} \]

\[ g = \{(x,a),(y,b),(z,b)\} \]

(a) Is $f$ injective?

(b) Is $f$ surjective?

(c) Is $g$ injective?

(d) Is $g$ surjective?

(e) Is $g \circ f$ surjective?

\item Let $A = \{a,b,c\}$, $B = \{1,2,3\}$, $C = \{x,y\}$. You are given functions $f:A\to B$ and $g:B\to C$.

\[ f = \{(a,1),(b,3),(c,2)\} \]

\[ g = \{(1,x),(2,y),(3,x)\} \]

(a) Is $f$ injective?

(b) Is $f$ surjective?

(c) Is $g$ injective?

(d) Is $g$ surjective?

(e) Is $g \circ f$ surjective?

\item Let $A = \{x,y,z\}$, $B = \{1,2,3\}$, $C = \{a,b,c\}$. You are given functions $f:A\to B$ and $g:B\to C$.

\[ f = \{(x,2),(y,1),(z,3)\} \]

\[ g = \{(1,a),(2,b),(3,c)\} \]

(a) Is $f$ injective?

(b) Is $f$ surjective?

(c) Is $g$ injective?

(d) Is $g$ surjective?

(e) Is $g \circ f$ surjective?

\end{enumerate}



\section{Graph theory}
\begin{enumerate}
\item Let $n\ge 3$. Recall the definition of cycles and complete graphs:
\[C_{n}=\{[n], E_{1}\}\]
\[K_{n}=\{[n], E_{2}\}\]
and define
\[G_{n}=\{[n], E_{2} \setminus E_{1}\}\]
\begin{itemize}
    \item Draw \(H, G_{4}, G_{5}, G_{6}, C_{5}, C_{6}, \overline{C_{i}}\)
    \item For all the above graphs, determine \(\Delta(G_{i}), \delta(G_{i}), \alpha(G_{i}), \omega(G_{i}), \chi(G_{i}), g(G_{i})\)
    \item Prove \((\forall i \geq 3) (G_{i} \simeq \overline{C_{i}})\)
\end{itemize}

\item Let \(G = ([n], E)\) be a graph.
\begin{itemize}
\item Prove: \(\chi (G) \geq \omega (G)\)
\item Prove: \(\chi (G) \geq \frac{n}{\alpha(G)}\)
\end{itemize}

\end{enumerate}
\



\end{document}
