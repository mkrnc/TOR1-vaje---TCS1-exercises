\begin{answer}{1.4}
The logical implication is:

\[
(A_1 \Rightarrow A_2) \land (A_1 \Rightarrow A_3) \Rightarrow (A_2 \Rightarrow A_3)
\]

For \( A_1(d) = 0 \), \( A_2(d) = 1 \), \( A_3(d) = 0 \), this implication is false. Therefore, the reasoning is incorrect.
\end{answer}
\begin{answer}{1.11}

Let us denote: $A$ – Arthur is a knight, $B$ – Bine is a knight, $C$ – Cene is a knight.

The following compound statement holds:

\[
(A \iff \neg C \lor \neg B) \land (B \iff C \land A)
\]

With the help of the truth table, we see that the statement is true only for the set $A = 1$ and $B = C = 0$.
\end{answer}
\begin{answer}{1.12}
Let \( A \) represent "A is a knight", etc. We seek the only solution \( d \) such that the following is true:

\[
A_1 \land B_1 \land C_1 \land D_1
\]

where:
\begin{align*}
A_1 &: A \iff (\neg D \land \neg C)\\
B_1 &: B \iff (\neg A \land \neg D \Rightarrow \neg C)\\
C_1 &: C \iff (\neg B \Rightarrow A)\\
D_1 &: D \iff (\neg E \Rightarrow \neg C \land \neg B)\\
\end{align*}


Since the truth table would contain 32 rows, we solve this by analyzing cases.

\subsection*{Case 1: \( A(d) = 1 \)}

Given \( A_1 \), \( D(d) = 0 \) and \( C(d) = 0 \). Substituting into \( C_1 \) with \( A(d) = 1 \) and \( C(d) = 0 \), we get:

\[
\neg (\neg B \Rightarrow 1) \Rightarrow \neg(B \lor 1) \Rightarrow \neg 1
\]

which is a false statement. Hence, this case is not possible.

\subsection*{Case 2: \( A(d) = 0 \)}

Given \( A_1 \), either \( C(d) = 1 \) or \( D(d) = 1 \).

\subsection*{Case 2.1: \( C(d) = 1 \)}

Since \( C_1 \), \( \neg B \Rightarrow 0 \), implies \( \neg B = 0 \), hence \( B(d) = 1 \).

Substituting into \( B_1 \) with \( A(d) = 0 \), \( B(d) = 1 \), and \( C(d) = 1 \), we get:

\[
1 \land \neg D \Rightarrow 0 \Rightarrow \neg D = 0
\]

thus, \( D(d) = 1 \).

Substituting into \( D_1 \), we get:

\[
\neg E \Rightarrow 0 \land 0
\]

Hence \( E(d) = 1 \).

\subsection*{Case 2.2: \( C(d) = 0 \) and \( D(d) = 1 \)}

From \( B_1 \), we get \( B(d) = 1 \), but the statement \( C_1 \) becomes false: \( 0 \iff (0 \Rightarrow 1) \).

Thus, \( B \), \( C \), \( D \), and \( E \) are knights, while \( A \) is a servant.

\end{answer}
\begin{answer}{1.27}

\begin{eqnarray*}
(A\Rightarrow B) \vee (B \Rightarrow C) &\Leftrightarrow & (\neg A \vee B) \vee (\neg B \vee C)\\
&\Leftrightarrow & \neg A \vee B \vee \neg B \vee C\\
&\Leftrightarrow & \neg A \vee (B \vee \neg B) \vee C\\
&\Leftrightarrow & \neg A \vee 1 \vee C\\
&\Leftrightarrow & 1.
\end{eqnarray*}
\end{answer}
\begin{answer}{1.33}
    We need to show that $x + \frac{1}{x} \geq 2$. Since $x \geq 0$, we can multiply the inequality by $x$ to get
    \[
    x^2 + 1 \geq 2x
    \]
    or equivalently,
    \[
    (x - 1)^2 \geq 0.
    \]
    This is obviously always true.
\end{answer}
\begin{answer}{1.34}
    Suppose there are only finitely many primes $p_1, p_2, \ldots, p_n$. Then the number $p = p_1 p_2 \cdots p_n + 1$ is not divisible by any prime $p_i$ and $p_i \neq p$ for each $i$. By definition, $p$ is therefore a prime number that is different from each of the previous ones. This is a contradiction.
\end{answer}
\begin{answer}{1.35}
    The opposite statement is: there exists a natural number greater than $1$.
\end{answer}
\begin{answer}{1.37}
We prove both directions separately:

($\Rightarrow$) Assume that $m$ and $n$ have different parities. Write $m = 2k$ and $n = 2l + 1$, substitute into the expression $m^2 - n^2$, and the result follows.

    ($\Leftarrow$) We show indirectly: If $m$ and $n$ have the same parity, then $m^2 - n^2$ is even. Consider both cases.
\end{answer}
\begin{answer}{1.39}
    \begin{enumerate}[label=(\roman*)]
        \item $(A \Rightarrow B) \wedge A \Rightarrow B$. This is true.
        \item  $(A \Rightarrow B) \wedge \neg A \Rightarrow \neg B$. This is not necessarily true.
        \item $((A \Rightarrow B) \wedge \neg A) \Rightarrow \neg B$. This is not necessarily true.
        \item $((A \Rightarrow B) \wedge (B \Rightarrow C)) \Rightarrow (\neg C \Rightarrow \neg A)$. This is true.
    \end{enumerate}
\end{answer}
\begin{answer}{2.2}
    \begin{enumerate}[label=(\roman*)]
        \item $P\cap S \neq \emptyset$
        \item $0 \in \mathbb{Z}\setminus \mathbb{N}$
        \item $\mathbb{N}\subseteq \mathbb{Z}$
        \item $\mathbb{Z}\nsubseteq \mathbb{N}$
        \item $P\setminus \{2\} \subseteq \overline{S}$
        \item $2\in S\cap P$
    \end{enumerate}
\end{answer}
\begin{answer}{2.4}
\begin{eqnarray*}
x\in (A\cup C)\cap (B\setminus C) &\Leftrightarrow & (x\in A \vee x\in C) \wedge (x\in B \wedge x\notin C)\\
 &\Leftrightarrow & ((x\in A \vee x\in C) \wedge (x\notin C))\wedge x\notin B\\
&\Leftrightarrow & ((x\in A \wedge x\notin C) \vee (x\in C \wedge x\notin C)) \wedge
 x\in B\\
&\Leftrightarrow & x\in A \wedge x\notin C  \wedge x\in B\\
&\Leftrightarrow & x\in A \wedge x\in B  \wedge x\notin C \\
&\Leftrightarrow & x \in (A\cap B)\setminus C.
\end{eqnarray*}
\end{answer}
\begin{answer}{2.5}
    Use direct proof.
\end{answer}
\begin{answer}{2.6}
    Prove separately each direction.
\end{answer}
\begin{answer}{2.7}

\begin{eqnarray*}
(x,y)\in A\times (B\cap C) &\Leftrightarrow & x \in A \wedge y\in B\cap C\\
&\Leftrightarrow & x \in A \wedge y\in B  \wedge y\in C\\
&\Leftrightarrow & x \in A \wedge x \in A\wedge y\in B  \wedge y\in C\\
&\Leftrightarrow & x \in A \wedge  y\in B  \wedge x \in A\wedge y\in C\\
&\Leftrightarrow & (x,y) \in A\times B \wedge  (x,y) \in A\times C\\
&\Leftrightarrow & (x,y) \in (A\times B)\cap   (A\times C).
\end{eqnarray*}
\end{answer}
\begin{answer}{2.8}

\begin{eqnarray*}
x\in (A\cap B )\setminus B  &\Leftrightarrow & x \in (A\cap B)  \wedge x\notin B\\
&\Leftrightarrow & (x\in A\wedge x\in  B ) \wedge x\notin B\\
&\Leftrightarrow & x\in A\wedge (x\in  B  \wedge x\notin B)\\
&\Leftrightarrow & x\in \emptyset.
\end{eqnarray*}
\end{answer}
\begin{answer}{2.10}
 (In slovene.)

\begin{enumerate}
    \item \textbf{Incorrect.} Take $A=\emptyset$, $B=\{\emptyset\}$, $C=\{\{\emptyset\}\}$.
    \item \textbf{Incorrect.} Take the same example as in (a).
    \item \textbf{Correct.} Proof by contradiction. Suppose the statement is not correct. Let $A\cap B\subseteq \overline{C}$, $A\cup C\subseteq B$ and let there exist $x\in A\cap C$. Thus, $x\in A$ and $x\in C$. Since $A\cup C\subseteq B$ by the second assumption, $x\in B$. It follows that $x\in A \cap B$. Since $A\cap B\subseteq \overline{C}$ by the first assumption, $x\in \overline{C}$. Contradiction, since $x\in C$.
    \item \textbf{Incorrect.} Take $A=C\neq B$.
    \item \textbf{Incorrect.} Take three pairwise disjoint non-empty sets.
\end{enumerate}

\end{answer}
\begin{answer}{2.21}
    \begin{enumerate}
\item[$\rightarrow$]
Let $A\subseteq B$.
We will show that $A\cap \overline{B} = \emptyset$ holds. By assumption, we have
\begin{equation}\label{eq:4}
(\forall x)(x\in A \Rightarrow x\in B).
\end{equation}
Suppose that there exists $x\in A\cap \overline{B}$. Then
\begin{eqnarray*}
x\in A\cap \overline{B} &\Rightarrow & x\in A \wedge x\in \overline{B}\\
						&\Rightarrow & x\in A \wedge x \notin B\\
						&\Rightarrow & x\in B \wedge x \notin B \quad (\textrm{due to } (\ref{eq:4})),
\end{eqnarray*}
a contradiction. Therefore, $A\cap \overline{B}=\emptyset$.

$\leftarrow$ Let $A\cap \overline{B}=\emptyset$. We will show that $A\subseteq B$. Take any $x\in A$. Then $x\notin \overline{B}$, since $A\cap \overline{B}=\emptyset$. Thus, $x\in B$. Since $x$ was arbitrary, it follows that $A\subseteq B$.

\item
\begin{eqnarray*}
A\setminus B & = & \{x\, ;\, x\in A \wedge x\notin B \}\\
             & = & \{x\, ;\, x\notin \overline{A} \wedge x  \in \overline{B} \}\\
             & = & \{x\, ; x\,  \in \overline{B} \wedge  x\notin \overline{A} \}\\
             & = & \overline{B} \setminus \overline{A}.
\end{eqnarray*}

\end{enumerate}


\end{answer}
\begin{answer}{3.4}
    \begin{enumerate}
    \item Domain: \(\{1, 2, 3\}\), Range: \(\{2, 3, 4\}\)
    \item Domain: \(\{1, 2, 3, 4\}\), Range: \(\{5, 3, 9\}\)
    \item Domain: \(\{1, 3, 4\}\), Range: \(\{2, 5\}\)
    \item Domain: \(\{-1, 2, 3\}\), Range: \(\{-1, 2, 3\}\)
    \item Domain: \(\{2, 9\}\), Range: \(\{0\}\)
\end{enumerate}

\end{answer}
\begin{answer}{3.6}
        For \(R_1 = \{(1, 2), (2, 3), (3, 4)\}\) and \(R_2 = \{(2, 4), (3, 5), (4, 6)\}\):
    \[
    R_2 \circ R_1 = \{(1, 4), (2, 5), (3, 6)\}, \quad R_1 \circ R_2 = \{(2, 4), (3, 5)\}.
    \]
    They are \textbf{not equal} because composition is not commutative.
    
\end{answer}
\begin{answer}{3.7}
    \begin{enumerate}
        \item \(R_2 \circ R_1 = \{(1, 6), (2, 4)\}\)
        \item \(R_2 \circ R_1 = \{(1, 4), (2, 6), (3, 4)\}\)
        \item \(R_2 \circ R_1 = \{(4, 5), (5, 6), (6, 4)\}\)
        \item \(R_1 \circ R_1 = \{(4, 4), (5, 5), (6, 6)\}\)
    \end{enumerate}
\end{answer}
\begin{answer}{3.8}
    \(R_1 = \{(x, y) \mid y = x + 1\}\), \(R_2 = \{(y, z) \mid z = y + 2\}\):
    \[
    R_2 \circ R_1 = \{(x, z) \mid z = x + 3, \, x, z \in \mathbb{Z}\}.
    \]
    
\end{answer}
\begin{answer}{3.9}
        For \(R = \{(1, 2), (2, 3), (3, 4)\}\):
    \[
    R \circ R = \{(1, 3), (2, 4)\}, \quad R \circ R \circ R = \{(1, 4)\},
    \]
    while \(R \circ R \circ R \circ R \circ R=\emptyset\)
    
\end{answer}
\begin{answer}{3.14}
        To check transitivity:
If \((x, y) \in R\) and \((y, z) \in R\), then \((x, z) \in R\).
Here, \((1, 2) \in R\) and \((2, 3) \in R\), but \((1, 3) \notin R\).
Thus, \(R\) is \textbf{not transitive}.

    
\end{answer}
\begin{answer}{3.45}
    \begin{enumerate}
        \item Using the definition of degree as the number of neighbors:
        \begin{itemize}
            \item $deg(1) = 3$ (connected to 2, 3, 4)
            \item $deg(2) = 2$ (connected to 1, 3)
            \item $deg(3) = 3$ (connected to 1, 2, 4)
            \item $deg(4) = 3$ (connected to 1, 3, 5)
            \item $deg(5) = 1$ (connected to 4)
        \end{itemize}
        \item
        \begin{itemize}
            \item $\Delta(G) = \max \{3, 2, 3, 3, 1\} = 3$.
            \item $\delta(G) = \min \{3, 2, 3, 3, 1\} = 1$.
        \end{itemize}
        \item The sum of degrees is $3+2+3+3+1 = 12$. The number of edges $|E(G)|$ is 6. The property states $\sum deg(v) = 2|E(G)|$. Since $12 = 2(6)$, the property holds.
    \end{enumerate}
\end{answer}
\begin{answer}{3.46}
    \begin{enumerate}
        \item Vertex removal removes the vertex and all incident edges. Removing vertex 1 removes edges $\{1,2\}$ and $\{1,4\}$. The remaining vertices are $\{2,3,4\}$ with edges $\{\{2,3\}, \{3,4\}\}$. This is isomorphic to the Path graph $P_3$.
        \item Edge removal removes only the edge, keeping vertices intact. Removing $\{1,2\}$ leaves vertices $\{1,2,3,4\}$ and edges $\{\{2,3\}, \{3,4\}, \{4,1\}\}$. This forms a path $2-3-4-1$ (which is $P_4$). Yes, it is still connected.
    \end{enumerate}
\end{answer}
\begin{answer}{3.47}
    \begin{enumerate}
        \item A complete graph has all possible edges. For $n=5$, size is $\binom{5}{2} = \frac{5 \times 4}{2} = 10$.
        \item
        \begin{itemize}
            \item The clique number $\omega(G)$ is the size of the largest complete subgraph. Since $K_5$ is complete, $\omega(K_5) = 5$.
            \item The independence number $\alpha(K_5)$ is the size of the largest empty induced subgraph. In a complete graph, no two vertices are non-adjacent, so $\alpha(K_5) = 1$.
        \end{itemize}
    \end{enumerate}
\end{answer}
\begin{answer}{3.48}
    \begin{enumerate}
        \item Yes. If we choose vertices $\{1, 2, 3\}$ from $C_5$, the induced edges are those in $C_5$ connecting these vertices: $\{1,2\}$ and $\{2,3\}$. The edge $\{1,3\}$ does not exist in $C_5$. Thus, the induced subgraph is $1-2-3$, which is $P_3$.
        \item $K_4$ contains all possible edges between 4 vertices.
        \begin{itemize}
            \item $C_4$ is a subgraph because we can select 4 vertices and remove the two "diagonal" edges.
            \item $C_4$ is \textbf{not} an induced subgraph. If we induce on any 4 vertices in $K_4$, we must keep \textit{all} edges between them. This yields $K_4$, not $C_4$.
        \end{itemize}
    \end{enumerate}
\end{answer}
\begin{answer}{3.49}
    \begin{enumerate}
        \item Both graphs have 6 vertices. Both are 2-regular (every vertex has degree 2).
        \item
        \begin{itemize}
            \item $g(G) = 6$ (the shortest cycle in $C_6$ is length 6).
            \item $g(H) = 3$ (the shortest cycles are the triangles $K_3$).
        \end{itemize}
        \item No, they are not isomorphic. While they share the same number of vertices and degree sequence, their structural parameters differ. Specifically, $g(G) \neq g(H)$. Also, $G$ is connected while $H$ is disconnected.
    \end{enumerate}
\end{answer}
\begin{answer}{3.50}
    \begin{enumerate}
        \item $deg(1)=1$, $deg(2)=2$, $deg(3)=1$.
        \item An automorphism must map a vertex to another vertex of the same degree.
        \begin{itemize}
            \item Vertex 2 (unique degree 2) must map to itself: $\phi(2)=2$.
            \item Vertices 1 and 3 (degree 1) can be swapped or stay fixed.
        \end{itemize}
        There are exactly 2 automorphisms:
        \begin{itemize}
            \item Identity: $\{1\to 1, 2\to 2, 3\to 3\}$
            \item Flip: $\{1\to 3, 2\to 2, 3\to 1\}$
        \end{itemize}
    \end{enumerate}
\end{answer}
\begin{answer}{3.51}
    \begin{enumerate}
        \item A graph with $\Delta(G)=2$ consists of paths and cycles. To have girth 4, it must contain a cycle of length 4. The graph $C_4$ satisfies this: every vertex has degree 2, and the cycle length is 4.
        \item By definition, a tree contains no cycles. Therefore, the set of cycle lengths is empty. Depending on convention, $g(Tree) = \infty$ or is undefined.
    \end{enumerate}
\end{answer}
\begin{answer}{3.4}
    Let $V(G) = \{v_1, ..., v_n\}$. The possible values for the degree of any vertex are integers in the set $\{0, 1, 2, ..., n-1\}$.

    However, it is impossible for a simple graph to contain both a vertex of degree $0$ (connected to no one) and a vertex of degree $n-1$ (connected to everyone else).
    \begin{itemize}
        \item If there is a vertex of degree $n-1$, no vertex can have degree $0$. The possible degree values are $\{1, 2, ..., n-1\}$.
        \item If there is a vertex of degree $0$, no vertex can have degree $n-1$. The possible degree values are $\{0, 1, ..., n-2\}$.
    \end{itemize}
    In either case, the vertices can take values from a set of size $n-1$. Since there are $n$ vertices (pigeons) and only $n-1$ possible degree values (holes), by the Pigeonhole Principle, at least two vertices must share the same degree.
\end{answer}
\begin{answer}{3.4}
    Consider a path of maximum length in $G$, denoted $P = x_0, x_1, ..., x_k$. Since the path is maximal, $x_k$ cannot be adjacent to any vertex outside the path (otherwise we could extend the path).

    However, we know $deg(x_k) \ge 2$. Therefore, $x_k$ must have at least two neighbors. Since these neighbors cannot be outside the path, they must be vertices already in the path (e.g., $x_i$ where $0 \le i < k-1$).

    If $x_k$ is connected to some $x_i$ in the path, the edges $(x_k, x_i)$ combined with the path segment $x_i, ..., x_k$ form a cycle. Thus, $G$ contains a cycle.
\end{answer}
\begin{answer}{3.4}
    \begin{enumerate}
        \item To maximize edges without creating a triangle, we should divide vertices into two sets and connect all vertices between the sets, but none within the sets (a complete bipartite graph). For $n=5$, we split vertices into sets of size 2 and 3. The graph $K_{2,3}$ has $2 \times 3 = 6$ edges. (Note: $C_5$ has only 5 edges).
        \item A bipartite graph divides $V(G)$ into two disjoint sets $A$ and $B$ where all edges go from $A$ to $B$. A triangle requires 3 vertices connected in a cycle $u-v-w-u$. In a bipartite graph, a path of length 3 ($u \in A \to v \in B \to w \in A$) ends back in set $A$. For $w$ to connect to $u$, there would need to be an edge within set $A$, which is forbidden. Thus, no odd cycles (including triangles) exist.
    \end{enumerate}
\end{answer}
\begin{answer}{3.4}
    \begin{enumerate}
        \item \textbf{$P_4$ ($1-2-3-4$):} The endpoints 1 and 4 have degree 1; the centers 2 and 3 have degree 2. An automorphism must map endpoints to endpoints. We can either fix everyone (Identity) or flip the whole path ($1 \leftrightarrow 4, 2 \leftrightarrow 3$). Total: 2 automorphisms.
        \item \textbf{$C_4$ ($1-2-3-4-1$):} Every vertex has degree 2, so we have more freedom. We can rotate the square ($0^\circ, 90^\circ, 180^\circ, 270^\circ$)—giving 4 rotations. We can also reflect it (flip horizontally, vertically, or diagonally). The automorphism group is the Dihedral group $D_4$, which has size $2n = 2(4) = 8$.
        \item Adding the edge $(1,4)$ to $P_4$ makes all vertices indistinguishable by degree (all become degree 2). This structural homogeneity allows any vertex to be mapped to any other vertex (vertex-transitivity), drastically increasing the number of symmetries from 2 to 8.
    \end{enumerate}
\end{answer}
\begin{answer}{3.60}
\begin{enumerate}
    \item Let $\rel{R}$ be a partial order on a set $A$. By definition, $\rel{R}$ is reflexive, anti-symmetric, and transitive. The definition of a preorder requires only that a relation be reflexive and transitive. Since $\rel{R}$ satisfies these two properties, $\rel{R}$ is a preorder.
    \item Let $A = \{a, b\}$ and let $\rel{R} = \{(a, a), (b, b), (a, b), (b, a)\}$.
    \begin{itemize}
        \item \textbf{Reflexivity:} $(a, a) \in \rel{R}$ and $(b, b) \in \rel{R}$, so it is reflexive.
        \item \textbf{Transitivity:} The only pairs that could fail transitivity are $(a, b)$ and $(b, a)$, which imply $(a, a) \in \rel{R}$ (satisfied); and $(b, a)$ and $(a, b)$, which imply $(b, b) \in \rel{R}$ (satisfied). So, it is transitive.
        \item \textbf{Anti-symmetry:} $(a, b) \in \rel{R}$ and $(b, a) \in \rel{R}$, but $a \ne b$. Thus, $\rel{R}$ is NOT anti-symmetric.
    \end{itemize}
    Since $\rel{R}$ is reflexive and transitive, it is a preorder. Since it is not anti-symmetric, it is not a partial order.
\end{enumerate}
\end{answer}
\begin{answer}{3.61}
\begin{enumerate}
    \item The standard additional property is \textbf{irreflexivity} (or anti-reflexivity).
    \item Proof:
    \begin{itemize}
        \item[($\Rightarrow$)] Let $\rel{R}$ be a strict partial order. By definition, a strict partial order is an irreflexive and transitive relation. (The property of anti-symmetry is redundant given irreflexivity and transitivity: if $(a, b) \in \rel{R}$ and $(b, a) \in \rel{R}$, then by transitivity, $(a, a) \in \rel{R}$. But since $\rel{R}$ is irreflexive, this is impossible. Thus, we cannot have both $(a, b)$ and $(b, a)$, which vacuously satisfies anti-symmetry.)
        \item[($\Leftarrow$)] Let $\rel{R}$ be a transitive and irreflexive relation. This is the standard definition of a strict partial order, so $\rel{R}$ is a strict partial order.
    \end{itemize}
    (Alternatively, if we define a strict partial order as an irreflexive, anti-symmetric, and transitive relation, we only need to show anti-symmetry follows from the other two, as noted above).
\end{enumerate}
\end{answer}
\begin{answer}{3.62}
Let $\rel{R}$ be a relation on $A$.
\begin{itemize}
    \item \textbf{From Linear Order $\rel{R}$ to Strict Linear Order $\rel{R}^-$:}
    Define $\rel{R}^- = \rel{R} \setminus \{(a, a) \mid a \in A\}$.
    If $\rel{R}$ is a linear order, it is reflexive, anti-symmetric, transitive, and total. $\rel{R}^-$ is clearly irreflexive by construction. It is transitive because $\rel{R}$ is transitive, and removing the reflexive pairs does not create any new transitivity failures. $\rel{R}^-$ is total for distinct elements: since $\rel{R}$ is total, for $a \ne b$, we have $(a, b) \in \rel{R}$ or $(b, a) \in \rel{R}$. Since $a \ne b$, these are not reflexive pairs, so $(a, b) \in \rel{R}^-$ or $(b, a) \in \rel{R}^-$. Thus, $\rel{R}^-$ is a strict linear order.

    \item \textbf{From Strict Linear Order $\rel{S}$ to Linear Order $\rel{S}^+$:}
    Define $\rel{S}^+ = \rel{S} \cup \{(a, a) \mid a \in A\}$.
    If $\rel{S}$ is a strict linear order, it is irreflexive, transitive, and total for distinct elements. $\rel{S}^+$ is clearly reflexive by construction. It is transitive: any transitivity failure must involve at least one new reflexive pair, say $(a, a) \in \rel{S}^+$ and $(a, b) \in \rel{S}^+$. This implies $(a, b) \in \rel{S}$, and thus $(a, b) \in \rel{S}^+$. Similarly for $(b, a) \in \rel{S}^+$ and $(a, a) \in \rel{S}^+$. It is anti-symmetric: if $(a, b) \in \rel{S}^+$ and $(b, a) \in \rel{S}^+$, by irreflexivity of $\rel{S}$, this is only possible if $a=b$. It is total: for any $a, b$, if $a=b$, $(a, a) \in \rel{S}^+$. If $a \ne b$, then $(a, b) \in \rel{S}$ or $(b, a) \in \rel{S}$, which means $(a, b) \in \rel{S}^+$ or $(b, a) \in \rel{S}^+$. Thus, $\rel{S}^+$ is a linear order.
\end{itemize}
Since $(\rel{R}^-)^+ = \rel{R}$ and $(\rel{S}^+)^- = \rel{S}$, the maps are inverses, establishing a one-to-one correspondence.
\end{answer}
