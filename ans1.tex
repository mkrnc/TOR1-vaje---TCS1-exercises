\begin{answer}{1.4}
The logical implication is:

\[
(A_1 \Rightarrow A_2) \land (A_1 \Rightarrow A_3) \Rightarrow (A_2 \Rightarrow A_3)
\]

For \( A_1(d) = 0 \), \( A_2(d) = 1 \), \( A_3(d) = 0 \), this implication is false. Therefore, the reasoning is incorrect.
\end{answer}
\begin{answer}{1.11}

Let us denote: $A$ – Arthur is a knight, $B$ – Bine is a knight, $C$ – Cene is a knight.

The following compound statement holds:

\[
(A \iff \neg C \lor \neg B) \land (B \iff C \land A)
\]

With the help of the truth table, we see that the statement is true only for the set $A = 1$ and $B = C = 0$.
\end{answer}
\begin{answer}{1.12}
Let \( A \) represent "A is a knight", etc. We seek the only solution \( d \) such that the following is true:

\[
A_1 \land B_1 \land C_1 \land D_1
\]

where:
\begin{align*}
A_1 &: A \iff (\neg D \land \neg C)\\
B_1 &: B \iff (\neg A \land \neg D \Rightarrow \neg C)\\
C_1 &: C \iff (\neg B \Rightarrow A)\\
D_1 &: D \iff (\neg E \Rightarrow \neg C \land \neg B)\\
\end{align*}


Since the truth table would contain 32 rows, we solve this by analyzing cases.

\subsection*{Case 1: \( A(d) = 1 \)}

Given \( A_1 \), \( D(d) = 0 \) and \( C(d) = 0 \). Substituting into \( C_1 \) with \( A(d) = 1 \) and \( C(d) = 0 \), we get:

\[
\neg (\neg B \Rightarrow 1) \Rightarrow \neg(B \lor 1) \Rightarrow \neg 1
\]

which is a false statement. Hence, this case is not possible.

\subsection*{Case 2: \( A(d) = 0 \)}

Given \( A_1 \), either \( C(d) = 1 \) or \( D(d) = 1 \).

\subsection*{Case 2.1: \( C(d) = 1 \)}

Since \( C_1 \), \( \neg B \Rightarrow 0 \), implies \( \neg B = 0 \), hence \( B(d) = 1 \).

Substituting into \( B_1 \) with \( A(d) = 0 \), \( B(d) = 1 \), and \( C(d) = 1 \), we get:

\[
1 \land \neg D \Rightarrow 0 \Rightarrow \neg D = 0
\]

thus, \( D(d) = 1 \).

Substituting into \( D_1 \), we get:

\[
\neg E \Rightarrow 0 \land 0
\]

Hence \( E(d) = 1 \).

\subsection*{Case 2.2: \( C(d) = 0 \) and \( D(d) = 1 \)}

From \( B_1 \), we get \( B(d) = 1 \), but the statement \( C_1 \) becomes false: \( 0 \iff (0 \Rightarrow 1) \).

Thus, \( B \), \( C \), \( D \), and \( E \) are knights, while \( A \) is a servant.

\end{answer}
\begin{answer}{1.27}

\begin{eqnarray*}
(A\Rightarrow B) \vee (B \Rightarrow C) &\Leftrightarrow & (\neg A \vee B) \vee (\neg B \vee C)\\
&\Leftrightarrow & \neg A \vee B \vee \neg B \vee C\\
&\Leftrightarrow & \neg A \vee (B \vee \neg B) \vee C\\
&\Leftrightarrow & \neg A \vee 1 \vee C\\
&\Leftrightarrow & 1.
\end{eqnarray*}
\end{answer}
\begin{answer}{1.34}
    We need to show that $x + \frac{1}{x} \geq 2$. Since $x \geq 0$, we can multiply the inequality by $x$ to get
    \[
    x^2 + 1 \geq 2x
    \]
    or equivalently,
    \[
    (x - 1)^2 \geq 0.
    \]
    This is obviously always true.
\end{answer}
\begin{answer}{1.35}
    Suppose there are only finitely many primes $p_1, p_2, \ldots, p_n$. Then the number $p = p_1 p_2 \cdots p_n + 1$ is not divisible by any prime $p_i$ and $p_i \neq p$ for each $i$. By definition, $p$ is therefore a prime number that is different from each of the previous ones. This is a contradiction.
\end{answer}
\begin{answer}{1.36}
    The opposite statement is: there exists a natural number greater than $1$.
\end{answer}
\begin{answer}{1.38}
We prove both directions separately:

($\Rightarrow$) Assume that $m$ and $n$ have different parities. Write $m = 2k$ and $n = 2l + 1$, substitute into the expression $m^2 - n^2$, and the result follows.

    ($\Leftarrow$) We show indirectly: If $m$ and $n$ have the same parity, then $m^2 - n^2$ is even. Consider both cases.
\end{answer}
\begin{answer}{1.40}
    \begin{enumerate}[label=(\roman*)]
        \item $(A \Rightarrow B) \wedge A \Rightarrow B$. This is true.
        \item  $(A \Rightarrow B) \wedge \neg A \Rightarrow \neg B$. This is not necessarily true.
        \item $((A \Rightarrow B) \wedge \neg A) \Rightarrow \neg B$. This is not necessarily true.
        \item $((A \Rightarrow B) \wedge (B \Rightarrow C)) \Rightarrow (\neg C \Rightarrow \neg A)$. This is true.
    \end{enumerate}
\end{answer}
\begin{answer}{2.2}
    \begin{enumerate}[label=(\roman*)]
        \item $P\cap S \neq \emptyset$
        \item $0 \in \mathbb{Z}\setminus \mathbb{N}$
        \item $\mathbb{N}\subseteq \mathbb{Z}$
        \item $\mathbb{Z}\nsubseteq \mathbb{N}$
        \item $P\setminus \{2\} \subseteq \overline{S}$
        \item $2\in S\cap P$
    \end{enumerate}
\end{answer}
\begin{answer}{2.4}
\begin{eqnarray*}
x\in (A\cup C)\cap (B\setminus C) &\Leftrightarrow & (x\in A \vee x\in C) \wedge (x\in B \wedge x\notin C)\\
 &\Leftrightarrow & ((x\in A \vee x\in C) \wedge (x\notin C))\wedge x\notin B\\
&\Leftrightarrow & ((x\in A \wedge x\notin C) \vee (x\in C \wedge x\notin C)) \wedge
 x\in B\\
&\Leftrightarrow & x\in A \wedge x\notin C  \wedge x\in B\\
&\Leftrightarrow & x\in A \wedge x\in B  \wedge x\notin C \\
&\Leftrightarrow & x \in (A\cap B)\setminus C.
\end{eqnarray*}
\end{answer}
\begin{answer}{2.5}
    Use direct proof.
\end{answer}
\begin{answer}{2.6}
    Prove separately each direction.
\end{answer}
\begin{answer}{2.7}

\begin{eqnarray*}
(x,y)\in A\times (B\cap C) &\Leftrightarrow & x \in A \wedge y\in B\cap C\\
&\Leftrightarrow & x \in A \wedge y\in B  \wedge y\in C\\
&\Leftrightarrow & x \in A \wedge x \in A\wedge y\in B  \wedge y\in C\\
&\Leftrightarrow & x \in A \wedge  y\in B  \wedge x \in A\wedge y\in C\\
&\Leftrightarrow & (x,y) \in A\times B \wedge  (x,y) \in A\times C\\
&\Leftrightarrow & (x,y) \in (A\times B)\cap   (A\times C).
\end{eqnarray*}
\end{answer}
\begin{answer}{2.8}

\begin{eqnarray*}
x\in (A\cap B )\setminus B  &\Leftrightarrow & x \in (A\cap B)  \wedge x\notin B\\
&\Leftrightarrow & (x\in A\wedge x\in  B ) \wedge x\notin B\\
&\Leftrightarrow & x\in A\wedge (x\in  B  \wedge x\notin B)\\
&\Leftrightarrow & x\in \emptyset.
\end{eqnarray*}
\end{answer}
\begin{answer}{2.10}
 (In slovene.)

\begin{enumerate}
    \item \textbf{Incorrect.} Take $A=\emptyset$, $B=\{\emptyset\}$, $C=\{\{\emptyset\}\}$.
    \item \textbf{Incorrect.} Take the same example as in (a).
    \item \textbf{Correct.} Proof by contradiction. Suppose the statement is not correct. Let $A\cap B\subseteq \overline{C}$, $A\cup C\subseteq B$ and let there exist $x\in A\cap C$. Thus, $x\in A$ and $x\in C$. Since $A\cup C\subseteq B$ by the second assumption, $x\in B$. It follows that $x\in A \cap B$. Since $A\cap B\subseteq \overline{C}$ by the first assumption, $x\in \overline{C}$. Contradiction, since $x\in C$.
    \item \textbf{Incorrect.} Take $A=C\neq B$.
    \item \textbf{Incorrect.} Take three pairwise disjoint non-empty sets.
\end{enumerate}

\end{answer}
\begin{answer}{2.21}
    \begin{enumerate}
\item[$\rightarrow$]
Let $A\subseteq B$.
We will show that $A\cap \overline{B} = \emptyset$ holds. By assumption, we have
\begin{equation}\label{eq:4}
(\forall x)(x\in A \Rightarrow x\in B).
\end{equation}
Suppose that there exists $x\in A\cap \overline{B}$. Then
\begin{eqnarray*}
x\in A\cap \overline{B} &\Rightarrow & x\in A \wedge x\in \overline{B}\\
						&\Rightarrow & x\in A \wedge x \notin B\\
						&\Rightarrow & x\in B \wedge x \notin B \quad (\textrm{due to } (\ref{eq:4})),
\end{eqnarray*}
a contradiction. Therefore, $A\cap \overline{B}=\emptyset$.

$\leftarrow$ Let $A\cap \overline{B}=\emptyset$. We will show that $A\subseteq B$. Take any $x\in A$. Then $x\notin \overline{B}$, since $A\cap \overline{B}=\emptyset$. Thus, $x\in B$. Since $x$ was arbitrary, it follows that $A\subseteq B$.

\item
\begin{eqnarray*}
A\setminus B & = & \{x\, ;\, x\in A \wedge x\notin B \}\\
             & = & \{x\, ;\, x\notin \overline{A} \wedge x  \in \overline{B} \}\\
             & = & \{x\, ; x\,  \in \overline{B} \wedge  x\notin \overline{A} \}\\
             & = & \overline{B} \setminus \overline{A}.
\end{eqnarray*}

\end{enumerate}


\end{answer}
\begin{answer}{3.4}
    \begin{enumerate}
    \item Domain: \(\{1, 2, 3\}\), Range: \(\{2, 3, 4\}\)
    \item Domain: \(\{1, 2, 3, 4\}\), Range: \(\{5, 3, 9\}\)
    \item Domain: \(\{1, 3, 4\}\), Range: \(\{2, 5\}\)
    \item Domain: \(\{-1, 2, 3\}\), Range: \(\{-1, 2, 3\}\)
    \item Domain: \(\{2, 9\}\), Range: \(\{0\}\)
\end{enumerate}

\end{answer}
\begin{answer}{3.6}
        For \(R_1 = \{(1, 2), (2, 3), (3, 4)\}\) and \(R_2 = \{(2, 4), (3, 5), (4, 6)\}\):
    \[
    R_2 \circ R_1 = \{(1, 4), (2, 5), (3, 6)\}, \quad R_1 \circ R_2 = \{(2, 4), (3, 5)\}.
    \]
    They are \textbf{not equal} because composition is not commutative.
    
\end{answer}
\begin{answer}{3.7}
    \begin{enumerate}
        \item \(R_2 \circ R_1 = \{(1, 6), (2, 4)\}\)
        \item \(R_2 \circ R_1 = \{(1, 4), (2, 6), (3, 4)\}\)
        \item \(R_2 \circ R_1 = \{(4, 5), (5, 6), (6, 4)\}\)
        \item \(R_1 \circ R_1 = \{(4, 4), (5, 5), (6, 6)\}\)
    \end{enumerate}
\end{answer}
\begin{answer}{3.8}
    \(R_1 = \{(x, y) \mid y = x + 1\}\), \(R_2 = \{(y, z) \mid z = y + 2\}\):
    \[
    R_2 \circ R_1 = \{(x, z) \mid z = x + 3, \, x, z \in \mathbb{Z}\}.
    \]
    
\end{answer}
\begin{answer}{3.9}
        For \(R = \{(1, 2), (2, 3), (3, 4)\}\):
    \[
    R \circ R = \{(1, 3), (2, 4)\}, \quad R \circ R \circ R = \{(1, 4)\},
    \]
    while \(R \circ R \circ R \circ R \circ R=\emptyset\)
    
\end{answer}
\begin{answer}{3.14}
        To check transitivity:
If \((x, y) \in R\) and \((y, z) \in R\), then \((x, z) \in R\).
Here, \((1, 2) \in R\) and \((2, 3) \in R\), but \((1, 3) \notin R\).
Thus, \(R\) is \textbf{not transitive}.

    
\end{answer}
